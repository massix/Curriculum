%%%%%%%%%%%%%%%%%%%%%%%%%%%%%%%%%%%%%%%%%
% Friggeri Resume/CV
% XeLaTeX Template
% Version 1.0 (5/5/13)
%
% This template has been downloaded from:
% http://www.LaTeXTemplates.com
%
% Original author:
% Adrien Friggeri (adrien@friggeri.net)
% https://github.com/afriggeri/CV
%
% License:
% CC BY-NC-SA 3.0 (http://creativecommons.org/licenses/by-nc-sa/3.0/)
%
% Important notes:
% This template needs to be compiled with XeLaTeX and the bibliography, if used,
% needs to be compiled with biber rather than bibtex.
%
%%%%%%%%%%%%%%%%%%%%%%%%%%%%%%%%%%%%%%%%%

\documentclass[]{friggeri-cv}

\begin{document}

\header{massimo}{gengarelli}{software developer} % Your name and current job title/field

%----------------------------------------------------------------------------------------
%	SIDEBAR SECTION
%----------------------------------------------------------------------------------------

\begin{aside} % In the aside, each new line forces a line break
  \section{contact}
  1260 avenue Jules Grec
  Antibes, Alpes-Maritimes
  06600 (France)
  ~
  +33 (0)6.43.87.57.67
  ~
  \href{mailto:massimo.gengarelli@gmail.com}{massimo.gengarelli@gmail.com}
  \href{https://github.com/massix/}{http://github.com/massix}
  \href{https://facebook.com/massimo.gengarelli}{facebook}
  \href{https://fr.linkedin.com/pub/massimo-gengarelli/46/375/468/}{linkedin}
  \section{languages}
  italian | native
  english | fluent
  french  | fluent
  \section{programming}
  C++, C, Clojure, Python,
  Vala, Java, Objective-C,
  Javascript, Typescript
  \section{environments}
  GNU/Linux, Mac OS X
  \section{softwares}
  gcc, leiningen, Emacs,
  VI(m), git, bzr, IntelliJ,
  Visual Studio Code
  \section{frameworks \& libraries}
  qt, gtk, springboot, angular,
  boost, vuejs
  \section{fields of interest}
  OSDev, low-level programming,
  networking, concurrency
\end{aside}

%----------------------------------------------------------------------------------------
%	EDUCATION SECTION
%----------------------------------------------------------------------------------------

\section{education}

\begin{entrylist}
  %------------------------------------------------
  \entry
      {2006--2011}
      {Bachelor of Science}
      {Universit\`a di Bologna ``Alma Mater Studiorum'', Italy}
      {\emph{ShockVM: an heterogeneous cluster of virtual machines for the Web}
        \\
        The thesis explored the idea of a remote spawning cluster of virtual
        machines entirely controlled through a web application. The technologies
        used were mainly based on KVM modules (\textbf{C}) and NoVNC.}
      %------------------------------------------------
\end{entrylist}

\begin{entrylist}
    \entry
      {2001--2006}
      {Technical Institute for Tourism}
      {Perito Turistico ``Marco Polo'' Rimini}
      {One of my passions has always been foreign languages so it was kind of
      natural for me to choose a school focused mainly on that.}
\end{entrylist}

%----------------------------------------------------------------------------------------
%	WORK EXPERIENCE SECTION
%----------------------------------------------------------------------------------------

\section{experience}

\begin{entrylist}
  \entry
      {2006--now}
      {Free Open Source Software Developer}
      {Everywhere, world}
      {\emph{Hobbyist developer \& hacker} \\
        Since the very beginning of my studies and even something before, I've
        always been a passionate developer and hacker.  I strive to stay active
        in the community and I have an (almost) infinite number of active personal
        projects.  One can see them all on my \href{https://github.com/massix/}{GitHub}
        personal page.
      }
      %------------------------------------------------
\end{entrylist}

\begin{entrylist}
  \entry
    {2017--now}
    {Alten Consultant for Orange}
    {Sophia Antipolis, France}
    {\emph{Djingo Software Engineer} \\
      In August 2017 I've joined the team ``Djingo'' as an expert developer.
      The project's goal is to become the new default vocal assistant for
      Orange customers in France, connected with all the other systems and
      services that Orange already delivers to its customers.  The team makes
      extensive use of the \textbf{Agile} development techniques, namely the
      \textbf{Scrum} technique. \\
      Skills: \textbf{Java}, \textbf{Springboot}, \textbf{Python}
    }
\end{entrylist}

\begin{entrylist}
  \entry
    {2016--2017}
    {Alten Consultant for Orange}
    {Sophia Antipolis, France}
    {\emph{Big Data Software Engineer \& OPS Assistant} \\
      I spent 7 months working as a developer on a newborn big data project driven by
      the DSIEP entity in Orange.  The project's goal was to collect all the
      data received by professional clients of Orange and retrieve useful datas
      from the users' interactions with the customers' care service.  During this
      project I realized that the infrastructure setup, mainly in terms of software
      development was insufficient to cover the needs of the project itself, so I
      spent some time setting up the development infrastructure.  I also developed
      some code on the Hadoop framework to collect and manipulate the data. \\
      Skills: \textbf{Clojure}, \textbf{Java}, \textbf{Hadoop}, \textbf{Gitlab},
      \textbf{Scala}
    }
\end{entrylist}


\begin{entrylist}
  \entry
    {2016}
    {Alten Consultant for Orange}
    {Sophia Antipolis, France}
    {\emph{Cloud Software Engineer} \\
      In my spare time during the experience within the métriKs product I had
      the pleasure to participate to the setup of the Cloud Environment for the
      DFY departement in Orange.  Together with a team of passionate developers
      and hackers we setup a \textbf{Continuous Integration \& Deployment}
      platform based on technologies such as \textbf{Docker}, \textbf{Gitlab} and
      \textbf{Artifactory}.  As a developer, I have also participated in the
      development of a component responsible of intercepting, filtering and handling
      the HTTP requests made by Docker clients and server in order to build an
      \textbf{ACL} service on top of it. \\
      Skills: \textbf{Python}, \textbf{Go}, \textbf{Docker}, \textbf{Ansible}
    }
\end{entrylist}

\begin{entrylist}
  \entry
      {2016}
      {Alten Consultant for Orange}
      {Sophia Antipolis, France}
      {\emph{métriKs Software Engineer} \\
        In January 2016 the Orange's search engine project has been closed
        and part of the developers has been relocated to a newly-born project
        responsible of collecting metrics and statistics from all the different
        projects in the department.  The projects made extensive use of a classic
        \textbf{ElasticSearch, Logstash, Kibana} (ELK) stack.  As a developer, I
        was responsible of developing the different micro-services which handled
        the data-collection and injection. \\
        Skills: \textbf{Javascript}, \textbf{Node.JS}
      }
      %-------------------------------------------------
\end{entrylist}

\begin{entrylist}
  \entry
      {2014--2016}
      {Alten Consultant for Orange}
      {Sophia Antipolis, France}
      {\emph{Backfill software Engineer} \\
        As part of the development team handling all the logic behind the
        Orange's search engine at \href{http://lemoteur.fr}{lemoteur.fr}, we're
        constantly improving our algorithms in order to provide the best answers
        for the users' queries. Our team covers all the logical processes from
        the crawling up to the creation of a responses' list for a given query. \\
        Skills: \textbf{C++11}, \textbf{NoSQL}, \textbf{TCL},
        \textbf{Boost.Asio}, \textbf{Chef}, \textbf{GNU Autotools},
        \textbf{Ubuntu}.
      }
      %------------------------------------------------
\end{entrylist}

\begin{entrylist}
  \entry
      {2012--2014}
      {Alten Consultant for Amadeus}
      {Sophia Antipolis, France}
      {\emph{Ticketing and Fare Elements} \\
        As a member of the Ticketing department in Amadeus, my team was
        responsible of decommissioning the old IBM platform (TPF) in favor of a
        distributed, modern one using a frontend/backend design with traffic
        balancing.
        \begin{itemize}
        \item \emph{Shadow Platform} a backend querying the old and the new
          platforms at the same time, producing offline statistics
        \item \emph{Customer's Server Interface} a backend capable to query a
          remote dislocated server providing real-time information based on the
          customer's needs.
        \end{itemize}
        Achievements: \emph{Development quality coordinator}, handling the
        quality of the development process within the context of the department
        I was working on. I've helped setting up some tools aiming at speeding
        and easing the whole deployment process. ``Agile Migration'': covered
        the role of \textbf{SCRUM master}. \\
        Skills: \textbf{C++}, \textbf{GNU/Linux}, \textbf{EDIFact}, \textbf{HLASM}, 
        \textbf{Boost}.
      }
      %------------------------------------------------
\end{entrylist}

\begin{entrylist}
  \entry
    {2011--2012}
    {Freelance developer at Assistenza Informatica Rimini}
    {Rimini, Italy}
    {\emph{Backend Junior Software Developer} \\
      On and off during the studies I worked as a Freelance developers for
      different companies in Italy.  The longest experience though has been
      with ``Assistenza Informatica'' for which I developed a backend service
      to handle the e-market transactions in their online shop.  The development
      process took 6 months. \\
      Skills: \textbf{Java}, \textbf{PHP}
    }
\end{entrylist}

%----------------------------------------------------------------------------------------
%	INTERESTS SECTION
%----------------------------------------------------------------------------------------

\section{interests}

\textbf{professional:} opensource, developing, new technologies, machine learning, OSDev, math \\
\textbf{personal:} my cats, my family, boardgames, wargames, rugby, movies, politics, economy


\end{document}
